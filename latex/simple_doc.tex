\documentclass[a4paper,12pt]{article} % добавить leqno в [] для нумерации слева

%%% Работа с русским языком
\usepackage{cmap} % поиск в PDF
\usepackage{mathtext} % русские буквы в формулах
\usepackage[T2A]{fontenc} % кодировка
\usepackage[utf8]{inputenc} % кодировка исходного текста
\usepackage[english,russian]{babel} % локализация и переносы

%%% Дополнительная работа с математикой
\usepackage{amsmath,amsfonts,amssymb,amsthm,mathtools} % AMS
\usepackage{icomma} % "Умная" запятая: $0,2$ —- число, $0, 2$ —- перечисление


%% Шрифты
\usepackage{euscript} % Шрифт Евклид
\usepackage{mathrsfs} % Красивый матшрифт

%% Свои команды
\DeclareMathOperator{\sgn}{\mathop{sgn}}

%% Перенос знаков в формулах (по Львовскому)
\newcommand*{\hm}[1]{#1\nobreak\discretionary{}
{\hbox{$\mathsurround=0pt #1$}}{}}

%%% Заголовок
\author{Сергей Чувакин }
\title{\LaTeX}
\date{\today}

\begin{document} % конец преамбулы, начало документа
	\maketitle  % быстро и дешево, красивый заголовок
	%\vspace{10cm}
Какой-то текст, просто текст

\section*{Заголовок} 

\subsection{Подзаголовок}

\subsubsection{Под-подзаголовок}
	
% \subsubsubsection{под-под-подзаголовок} % дальше уже нельзя 


\section{Математика}

есть два вида формул сторочные и выносные
строчные: здесь какой то текст $x^2 = 5$ дальше текст 


выносные: 

текст текст $$\frac{x}{x^3} = 10$$

\subsection{стандартные функции}

$\sin x = 5$
$$\ln x = 56\\$$

\subsection{symbols}
$$2\times2 \ne 5$$
$2\cdot2$
$A\cap B$, $A\cup B$

\textbf{\Huge{miktex}}

\emph{ffffff}

{\em   ggggg}

\subsection{диакретические знаки}

\[\bar x=8\]
\[\tilde{axb} = 9\]
\[\overline{axb} = 9\]
\[\widetilde{axb} = 9\]
\subsection{буквы других алфавитов}
$\alpha,\tg$ 
$A, \tg$ 

\section{формулы в несколько строк} 
\subsection{очень длинные формулы}

\begin{multline}
1+2+3+5+6+7+\dots+\\+50+51+52+53+54+55+\dots+\\+70+71+72+73+74+75+\dots+\\+95+96+97+98+99+100  =  5050 \tag{S}\label{SUM}
\end{multline}

\subsection{несколько формул}


\begin{align} \label{33}
2\times 2 &= 4 &6\times 8 &= 48 & 2\times 2 = 4\\ 
3\times 3  &=9 &5\times 5 &= 25 & 2\times 2444 = 4\\
10 \times 666 &= 6660 &\frac {3}{2} &= 1.5 & 211\times 2 = 4
\end{align}

\begin{equation}\label{22}
\begin{aligned}
2\times 2 &= 4 &6\times 8 &= 48 & 2\times 2 = 4\\
3\times 3  &=9 &5\times 5 &= 25 & 2\times 2444 = 4\\
10 \times 666 &= 6660 &\frac {3}{2} &= 1.5 & 211\times 2 = 4	
\end{aligned}
\end{equation}
здесь последнее уравнение \% \eqref{22}, а здесь предпоследнее \eqref{33} до этой строчки 

\subsection{система уравнений}
\[\left\{
\begin{aligned}
2\times 2 &= 4 &6\times 8 &= 48 & 2\times 2 = 4\\
3\times 3  &=9 &5\times 5 &= 25 & 2\times 2444 = 4\\
10 \times 666 &= 6660 &\frac {3}{2} &= 1.5 & 211\times 2 = 4	
\end{aligned}\right.
\]

\tableofcontents


\end{document}